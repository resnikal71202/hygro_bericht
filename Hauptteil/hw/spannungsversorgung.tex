\subsubsection{Spannungsversorgung}
Da auf dem Arduino die Versorgungsspannung per USB eingespeist wird, wir jedoch ein Netzteil mit 12V angeschlossen haben, mussten wir die Spannung zwei mal auf kleinere Spannungswerte drosseln. Dies geschieht mithilfe zweier Microcontroller, die im Schaltplan mit U1 und U2 gekennzeichnet sind. Die 12V der Spannungsversorgung sind nötig, weil der Sender eine Versorgungsspannung von 12V hat und es einerseits einfacher ist, Spannungen herunterzutransformieren und zum anderen ist es in der Basis des Systems nicht unbedingt notwendig Energie zu sparen, da dieses sich per Netzteil versorgt und keine Rücksicht auf Laufzeiten von Batterien genommen werden musste.\\
Die LEDs sollen eine intakte Versorgung mit der jeweiligen Spannung darstellen. Für eine durchgehende Versorgung ohne Schwingungen dienen die zwei Kondensatoren am Ausgang des ICs. 
