\subsubsection{Stromlaufplan}
Nachdem alle benötigten Bauteile identifiziert waren, wurde der Blockschaltplan gezeichnet.\\ 
Nach dem Erstellen eines Blockschaltplanes und der Software für den Prototypen, ging es nun an das Erstellen des Schaltplanes und die Entflechtung der Boarddatei für die Platine. Hierfür wurde die CAD-Software Eagle benutzt, da in den Bibliotheken dieser Software und auch im Internet für fast alle erdenklichen Bauteile bereits die sogenannten Footprints für die Board- als auch die Schaltplan-Symbole für die Schaltplandatei zu finden sind. Hierfür mussten diverse Anschlusspins des Microcontrollers neu belegt werden, da Sender und Empfänger auf unterschiedlichen Pins kommunizieren sollten, und die Software bereits so geschrieben war. 
