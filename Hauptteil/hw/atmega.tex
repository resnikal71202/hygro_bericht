\subsubsection{Der Microcontroller Atmega328}
Der verbaute Atmega 328 verfügt über 16 Pins und hat einige Vorteile gegenüber vergleichbaren ICs. Er ist einfach aufgebaut, einfach zu programmieren, günstig und zählt zu den sogenannten „Low-Power-Microcontrollern“. …(mehr Infos) \\
Da der Atmega328 einen externen Quarz braucht, um genau zu funktionieren, musste ein passender Quarz gesucht werden. Je höher der Takt, desto größer ist auch sein Stromverbrauch. Da der Microcontroller jedoch mit höherer Frequenz auch schneller arbeiten kann, wurde ein Quarz mit 16 Hz bei diesem Projekt verwendet. 
