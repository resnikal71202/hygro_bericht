\subsection{Satelit}

\subsubsection{Versorgungsspannung}
Um die Versorgunsspannung zu messen wird die interne $1.1V$ interne Spannungsreferenz des ATmega verwendet. Der verwendete ADC verwendet als maximal Spannung die Versorgungsspannung. Diese Entspricht also dem Maximum von 1024 (10 Bit). Durch Messung der Internen Referenzspannung können dann Rückschlüsse auf die Versorgungsspannung getroffen werden. 
z.B. ergibt die Messung der Referenz einen Wert von $350$, so kann durch die Formel $\frac{1.1V}{ADC Result} * 1024$ berechnet werden. Siehe \ref{satelit spannung} Zeile 8.
\subsubsection{low power}
Die Anforderungen an die Laufzeit wurden durch ein \grqq deep-sleep \grqq realisiert. Hierfür wurde auf das git von Rocketscream zurückgegriffen\cite{lowpower}. Hier kann eine maximale sleep Dauer von $8s$ eingestellt werden. Für größere Zeiten muss ein Externer Interrupt festgelegt werden. Um eine weitere externe Schaltung zu vermeiden wird hier die Clock abgeschalten nicht abgeschaltet. Diese wird verwendet um einen Timer zu steuern, welcher nach $8s$ einen Interrupt erzeugt. Mit diesem Wacht der ATmega auf. Dies wird wiederholt um größere Zeiten zu erreichen. Siehe \ref{lowpowersleep}
\subsubsection{eeh210}

\subsubsection{cbp Realisierung}
Die cbp Realisierung basiert darauf, dass sämtliche Daten sich durch, z.B. \mintinline{c++}{dtostrf();}, zu einem String formatieren lassen. Diese Teilstrings werden anschließend durch Kommas getrennt aneinandergereiht. Hierfür wird hauptsächlich die funktion \mintinline{c++}{strcat();} verwendet. Siehe \ref{cbp satelit}