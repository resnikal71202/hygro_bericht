\section{Technische Rahmenbedingungen}
Das Produkt wird aus 3 Einheiten bestehen. Eine davon wird eine Basisstation sein, die anderen 2 werden Satelliten sein.\\Jede Einheit wird über einen Sender und einen Empfänger verfügen. Mit diesen wird eine 2- Richtungskommunikation realisiert. An jedem Satelliten wird ein Feuchtigkeitssensor verbaut. Diese Feuchtigkeitssensoren werden mit einem Mikrocontroller verbunden. Ziel ist es zuerst eine Grundfunktion herzustellen, im Nachhinein soll es eine Möglichkeit geben weitere Sensoren hinzuzufügen, Beispielsweise einen Lichtsensor, einen Luftdrucksensor, Bodenfeuchte oder einen CO2- Sensor. Die Satelliten werden mit Hilfe eines Akkus versorgt, dabei wird die Spannung überwacht. Außerdem werden wir Schutzfunktionen für Überlast und Kurzschlüsse hinzugefügt.  Die Basis wird über Relais verfügen. Hiermit kann dann eine Abluft / Zuluft geschaltet werden. Mittels 2 Tastern kann man die minimale und maximale Luftfeuchtigkeit anpassen. Dies passiert in einer Vorgegebenen Schrittweite. Des Weiteren soll es möglich sein die aktuellen Daten an einem Display auszulesen. 
